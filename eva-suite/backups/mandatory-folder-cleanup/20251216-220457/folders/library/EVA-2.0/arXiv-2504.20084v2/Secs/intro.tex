\section{Introduction}
\label{sec:intro}

Recently, the rapid acceleration of large language model (LLM) development has transformed artificial intelligence (AI) from a narrow, task-specific paradigm into a general-purpose intelligence with far-reaching implications. Contemporary LLMs demonstrate increasingly sophisticated linguistic, reasoning, and problem-solving capabilities, and are showcasing superb human-like behaviors, prompting a fundamental research question \citep{scott2023you, keeling2024can}: 

\begin{quote}
\begin{center}
\emph{To what extent do these systems exhibit forms of awareness?}
\end{center}
\end{quote}

Here, it is crucial to clarify that while the concept of \textit{AI consciousness} remains philosophically contentious and empirically elusive, the concept of \textit{AI awareness}\textemdash defined as a system’s functional capacity to represent and reason about its own states, capabilities, and the surrounding environment\textemdash has become an important and measurable research frontier, \ie, \autoref{fig:google_trend} demonstrates that the recent focus on AI awareness is growing, even surpassing AI consciousness.

\begin{figure}[tb]
    \centering
    \includegraphics[width=\linewidth]{Figs/google_trend.png} 
    \caption{Google Trends search interest (normalized 0–100) for the terms ``AI awareness'' (\textcolor{myred}{red}) and ``AI consciousness'' (\textcolor{myblue}{blue}) over the past five years (31 May 2020 – 30 May 2025). While both topics show gradual growth, the red line accelerates markedly from late 2023 onward, eventually overtaking the blue line and highlighting the rising public focus on functional, measurable aspects of AI's cognition}
    \label{fig:google_trend}
\end{figure}

Awareness, as conceptualized in cognitive science and psychology, typically encompasses four distinct yet interrelated dimensions: 
\begin{itemize}
\item \textbf{Metacognition:} ability to monitor and regulate cognitive processes \citep{flavell1979metacognition}.
\item \textbf{Self-Awareness:} recognizing and representing one's identity and limitations \citep{duval1972objective}.
\item \textbf{Social Awareness:} capacity to interpret others' mental states and intentions \citep{lieberman2007social}.
\item \textbf{Situational Awareness:} maintaining an accurate representation of the external environment and anticipating future states \citep{endsley1995toward}.
\end{itemize}

Recent computational cognitive science research indicates that certain aspects of these awareness dimensions can be approximated by LLMs through metacognitive behaviors \citep{Didolkar2024, renze2024self}, calibrated epistemic confidence \citep{steyvers2025large}, and perspective-taking tasks \citep{wilf2024think}. These emergent functional abilities highlight important questions regarding how awareness manifests within LLMs, how it might be systematically assessed, and its implications for AI capabilities, safety, and alignment.

Despite increasing scholarly interest, research on AI awareness remains fragmented across disciplines, with limited consensus on definitions, methodologies, and broader implications. While some researchers point to emergent behaviors revealed through introspection tasks \citep{huanglarge} or theory-of-mind (ToM)-inspired evaluations \citep{Kosinski2024}, others caution against anthropomorphic interpretations of statistical model outputs, arguing that apparent self-awareness could result from sophisticated pattern recognition rather than genuine metacognitive representation \citep{van2023chatgpt, shanahan2024talking}. Furthermore, current methods for assessing awareness in AI often face challenges such as prompt sensitivity, data contamination, and insufficient robustness across varying contexts.

Existing literature has laid important groundwork on closely related concepts. For instance, \citet{butlin2023consciousness} provided the first systematic account of theoretical foundations and potential prerequisites for consciousness in artificial intelligence. Similarly, \citet{ward2025towards} explored agency, theory of mind, and self-awareness as foundational criteria for considering AI as possessing personhood. Additionally, \citet{metzinger2021artificial} addressed ethical and philosophical questions surrounding the construction of artificial consciousness and self-modeling systems. Differing from these foundational works, our review specifically synthesizes and advances understanding of AI awareness as a distinct, functional, and measurable construct, separate from consciousness or personhood.

\begin{figure}[tb]
    \centering
    \includegraphics[width=\linewidth]{Figs/Roadmap.png} 
    \caption{The roadmap of our review}
    \label{fig:roadmap}
\end{figure}

This review provides a comprehensive, cross-disciplinary synthesis of AI awareness research. As illustrated in \autoref{fig:roadmap}, we first establish a clear theoretical framework, differentiating AI awareness explicitly from AI consciousness, and examining how awareness-related concepts have been formalized across cognitive and computational sciences. We then critically analyze existing experimental methods for evaluating AI awareness, emphasizing empirical results and highlighting methodological shortcomings. Subsequently, we explore how functional awareness might positively influence AI capabilities, including enhanced reasoning, planning, and safety improvements. Finally, we address the emerging risks associated with increasingly aware AI systems, particularly concerns highlighted within the AI safety and alignment communities\textemdash such as deception, manipulation, emergent uncontrollability\textemdash and ethical challenges, including false anthropomorphism.

By integrating insights from artificial intelligence, cognitive science, psychology, and AI safety, this review aims to deliver a structured and comprehensive perspective on current knowledge and outline future research trajectories. Ultimately, we seek to deepen understanding of one of the most significant interdisciplinary challenges at the nexus of AI, cognitive science, and societal implications.

Overall, our key contributions are as follows:
\begin{itemize}
\item We introduce a novel framework defining four principal dimensions of AI awareness: metacognition, self-awareness, social awareness, and situational awareness.
\item We systematically summarize existing methods, significant findings, and critical limitations in evaluating AI awareness, thereby laying the foundations for robust, evergreen evaluation practices.
\item We provide the first structured analysis categorizing how enhanced AI awareness contributes positively to capabilities and simultaneously escalates associated risks. By clarifying that AI awareness functions as a double-edged sword, we emphasize the importance of cautious and guided development.
\end{itemize}

\mytcolorbox{Decoding the intricate relationship between awareness and capability is key to the next era of artificial intelligence\textemdash offering opportunities for innovation, but demanding careful navigation of emergent risks and responsibilities.}
